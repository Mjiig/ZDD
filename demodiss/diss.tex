% Template for a Computer Science Tripos Part II project dissertation
\documentclass[12pt,a4paper,twoside,openright]{report}
\usepackage[pdfborder={0 0 0}]{hyperref}    % turns references into hyperlinks
\usepackage[margin=25mm]{geometry}  % adjusts page layout
\usepackage{graphicx}  % allows inclusion of PDF, PNG and JPG images
\usepackage{verbatim}
\usepackage{docmute}   % only needed to allow inclusion of proposal.tex
\usepackage{amsmath}
\usepackage{amssymb}
\usepackage[labelformat=simple]{subcaption}

\renewcommand\thesubfigure{(\alph{subfigure})}

\raggedbottom                           % try to avoid widows and orphans
\sloppy
\clubpenalty1000%
\widowpenalty1000%

\renewcommand{\baselinestretch}{1.1}    % adjust line spacing to make
                                        % more readable

\begin{document}

\bibliographystyle{plain}


%%%%%%%%%%%%%%%%%%%%%%%%%%%%%%%%%%%%%%%%%%%%%%%%%%%%%%%%%%%%%%%%%%%%%%%%
% Title


\pagestyle{empty}

\rightline{\LARGE \textbf{Angus Hammond}}

\vspace*{60mm}
\begin{center}
\Huge
\textbf{Representing Solutions to the Travelling Salesman Problem Using Zero Suppressed Binary Decision Diagrams} \\[5mm]
Computer Science Tripos -- Part II \\[5mm]
Corpus Christi College \\[5mm]
\today  % today's date
\end{center}

%%%%%%%%%%%%%%%%%%%%%%%%%%%%%%%%%%%%%%%%%%%%%%%%%%%%%%%%%%%%%%%%%%%%%%%%%%%%%%
% Proforma, table of contents and list of figures

\pagestyle{plain}

\chapter*{Proforma}

{\large
\begin{tabular}{ll}
Name:               & \bf Angus Hammond                       \\
College:            & \bf Corpus Christi College                     \\
Project Title:      & \bf Representing Solutions to the Travelling Salesman Problem Using Zero Suppressed Binary Decision Diagrams \\
Examination:        & \bf Computer Science Tripos -- Part II, July 2017  \\
Word Count:         & \bf ????  \\
Project Originator: & Angus Hammond                    \\
Supervisor:         & Dr Timothy Griffin                    \\ 
\end{tabular}
}
\stepcounter{footnote}


\section*{Original Aims of the Project}

To build an implementation of Zero Suppressed Binary Decision Diagrams (ZDDs), and a solver for the Travelling Salesman Problem (TSP) that makes use of them to represent possible solutions. The performance of this solver was then compared to that of a standard dynamic programming solution to the same problem, with the intention to show that both had the same asymptotic complexity. 


\section*{Work Completed}

An python package implementing ZDDs, as well as a number of standard logical operators on them has been implemented and tested. A TSP solver built on this package has been built, along with a solver using a dynamic programming algorithm and a brute force solver (for correctness verification). In order to measure the performance of these solvers, a framework for generating random instance of TSP has been built, and the performance of all of the solvers on these instances has been measured.

\section*{Special Difficulties}

A mistaken assumption about how operations that can be used to encode negation behave on ZDDs (due to a substantial difference to the behaviour of traditional BDDs), led to a significant delay in completing the correct implementation of the implication and negation operators, which are essential for this project.
 
\newpage
\section*{Declaration}

I, Angus Hammond of Corpus Christi College, being a candidate for Part II of the Computer
Science Tripos, hereby declare
that this dissertation and the work described in it are my own work,
unaided except as may be specified below, and that the dissertation
does not contain material that has already been used to any substantial
extent for a comparable purpose.

\bigskip
\leftline{Signed [signature]}

\medskip
\leftline{Date [date]}

\tableofcontents

\listoffigures

%%%%%%%%%%%%%%%%%%%%%%%%%%%%%%%%%%%%%%%%%%%%%%%%%%%%%%%%%%%%%%%%%%%%%%%
% now for the chapters

\pagestyle{headings}

\chapter{Introduction}
\section{Project Outline}
This project seeks to demonstrate an equivalence between two different algorithms for solving the Travelling Salesman Problem, an obvious dynamic programming solution, and a solution based on techniques from logic theory that represents all Hamiltonian paths of a graph and enumerates them in a structured way. The second algorithm represents Hamiltonian paths of a graph as a constrained set of propositional variables and stores the set of satisfying assignments using a data structure called a Zero-suppressed Binary Decision Diagram (ZDD). This data structure is a variant of the better known Binary Decision Diagram (BDD). The theoretical basis for believing this equivalence exists, an explanation of the origins and working of ZDDs and descriptions of both of the algorithms are included in the Preparation chapter.

The majority of the code I have written as part of this project is a library in Python for building and manipulating ZDDs efficiently, including support for producing a new ZDD from a previously built one using several standard propositional operations (for example negation, conjunction, disjunction and implication). In addition I have implemented both of the algorithms being reviewed, along with a third simpler algorithm for the Travelling Salesman Problem to help verify the correctness of both implementations. In order to test all of the implementations I have written two procedures to generate instances of the Travelling Salesman Problem, one of which produces problem instances with analytically verifiable solutions, to further aid verification of the correctness of all the algorithms, with the other producing more general problem instances in order to demonstrate the generality of all the algorithms. How each of these components is implemented is described in more detail in the Implementation chapter.

In order to experimentally demonstrate the equivalence of the two algorithms I have run them on a large number of automatically generated instances of the Travelling Salesman Problem and measured both the average time and average number of logical operations taken by the two algorithms to produce a solution, as a function of the number of cities in the problem. In both cases the function is expected to have the form $\Theta(2^nn^2)$.

\chapter{Preparation}

\section{The Travelling Salesman Problem}
The Travelling Salesman Problem exists as both a function problem and a decision problem. In this project I have focused on the function problem, but it is trivial to construct a program solving the decision problem given a program solving the function problem. The problem is to find the lowest cost path that visits all of a set of cities. More formally, given a set of cities $C$ and a cost function $f: C\to\mathbb{R}$ find the minimum possible value of $\sum_{i=1}^{|C|-1}f(x_i,x_{i+1})$ where each $x_i$ has been assigned a distinct value from $C$.

The related decision problem takes a target distance as well as a set of cities and a cost function, and is only required to evaluate whether there is a path with total cost less than the given target distance. This form of the problem is known to be NP-Complete.

Variants of the problem exist, for example requiring that the path start and finish in the same city, not guaranteeing that the the cost function be total (i.e.\ prohibiting certain cities from being visited consecutively) or guaranteeing that the cost function will only take integer values. Most such variant are easy to reduce to an instance of the described function problem.

\section{Solving The Travelling Salesman Problem with Dynamic Programming} \label{dynamicprogramming}
An entirely naive brute force approach to the Travelling Salesman Problem takes $\Theta(n!)$ time to run. This can be improved on significantly with dynamic programming. Let the $C$ be the set of cities and $f$ be the cost function. Define $g(S, c)$, where $S\subseteq C$ and $c\in S$ to be the lowest possible cost of a path that starts in $c$ and visits all the cities in $S$. Computing $g({c}, c)$ for all $c\in C$ is trivial, since paths containing only a single city have a cost of zero. Then for all other values of $S\subseteq C$, compute $g(S,c)$ as 

$$
g(S,c)=\min_{c'\in S\setminus\{c\}}(g(S\setminus\{c\},c') + f(c,c'))
$$

Once all values of $g$ have been computed, the solution to the problem is given by

$$
\min_{c\in C} g(C,c)
$$

If $C$ has size $n$ there are $2^n$ subsets of $C$, so $n2^n$ possible sets of arguments to $g$, each of which requires evaluation of up to $n$ terms to compute, this algorithm has a time complexity of at most $\mathcal{O}(n^22^n)$.

\section{Hamiltonian Paths as Propositional Sentences} \label{hamiltonianpath}
An alternative approach to solving the Travelling Salesman Problem is to attempt to enumerate all of the Hamiltonian paths through the complete graph that represents each city as a node. For each Hamiltonian path we calculate the cost of the path it represents and return the lowest observed value. Although the Hamiltonian paths through a complete graph are easy to generate directly, doing so leads to a very slow algorithm for the Travelling Salesman Problem ($\Theta(n!)$). Instead we represent the set of Hamiltonian paths using a constrained set of propositional variables, which will subsequently allow us to enumerate them in a more structured way.

Given a set $C$ of cities, define a set of variables $visited_{c,i}$ for all $c\in C$ and $1\leq i\leq|C|$. An assignment to this set of variables is interpreted as encoding a path that visits city $c$ as the $i$th city in the path if and only if $visited_{c,i}$ is true. In order to ensure that only assignments corresponding to valid Hamiltonian paths are possible the following constraints can be imposed:

\begin{gather*}
visited_{c,i}\implies\bigwedge_{c'\in C\setminus c}\neg visited_{c',i} \\ \\
visited_{c,i}\implies\bigwedge_{1\leq i'\leq |C|,i'\neq i}\neg visited_{c,i'} \\ \\
\bigwedge_{c\in C}\bigvee_{1\leq i\leq |C|} visited_{c,i}
\end{gather*}

Intuitively speaking, the first constraint requires that at most a single city is visited at a time, the second constraint requires that each city is visited at most once and the third constraint requires that every city is visited at some point. Since the number of cities to visit and the number of opportunities for the path to visit a city are exactly equal, this implies the constraint that a city is visited at every step. In order to enumerate Hamiltonian paths of an arbitrary graph, additional constraints could be added requiring that when a given city is visited at a given point along the path a city which is connected to it by an arc in the graph is visited at the next point in the path. However in the special case I have considered in this project of representing Hamiltonian paths only in order to solve the Travelling Salesman Problem this is unnecessary because the graph of connected cities is generally complete.

\section{Binary Decision Diagrams}
Since all permutations of the nodes are Hamiltonian paths of a complete graph, there will be $n!$ possible Hamiltonian paths. Therefore in order to use an algorithm based on enumerating them to solve the Travelling Salesman Problem it is necessary to find an algorithm for propositional satisfiability that enumerates satisfying assignments in a structured way, so that computations can be shared between them. To achieve this I have used ZDDs, which are a type of decision diagram designed to most efficiently represent the satisfying assignments for a set of propositional constraints when there are few satisfying assignments and few asserted propositional letters in the satisfying assignments.

The simplest data structure in the family of decision diagrams is a binary decision tree, which represents propositional sentences as a rooted tree in which internal nodes are labelled with propositional letters and leaves are labelled with $0$ or $1$. Every node at the same level of the tree is labelled with the same propositional letter, so that all paths from the root of the tree to a leaf pass through exactly one node labelled with each propositional letter, and every such path passes through nodes labelled with the propositional letters in the same order. Each internal node has exactly two children, connected to it by arcs labelled $T$ and $F$. A path from the root to a leaf of the tree corresponds to an assignment, where a propositional letter is asserted in the corresponding assignment if and only if the path follows the arc labelled $T$ out of a node labelled with the propositional letter (and therefore not asserted if and only if the path follows the arc labelled $F$ out of a node labelled with it). Since leaf nodes obviously correspond to paths in a rooted tree, assignments to the propositional letters also correspond to leaf nodes of the tree. Therefore we can use a tree to represent a propositional sentence by labelling a leaf node $1$ if the assignment to the propositional letters is corresponds to satisfies the propositional constraints, and label a leaf node $0$ otherwise.

Binary decision trees give a canonical representation of propositional sentences, in that two propositional sentences are represented by the same binary decision tree if and only if they have the exact same set of satisfying assignments. An example of a a binary decision tree is shown in Figure \ref{decisiontree} which represents the propositional sentence, 

$$
(p\wedge\neg q\wedge\neg r) \vee (\neg p\wedge q\wedge r) \vee (\neg p\wedge\neg q\wedge\neg r)
$$

shown in full disjunctive normal form as each conjunctive clause corresponds to a path from the root of the tree to a leaf labelled $1$, called a $1$-path. In the shown diagram, arcs labelled $T$ are represented by solid lines, and arcs labelled $F$ are represented by dashed lines, which is a standard notation that I will use for all types of decision diagram.

\begin{figure}[tbh]
\centering
\includegraphics{{figs/tree.dot}.pdf}
\caption{A binary decision tree representing a propositional sentence with three satisfying assignments. Solid lines represent arcs labelled $T$, dashed lines represent arcs labelled $F$, internal nodes are elliptical and labelled with propositional letters and leaf nodes are square and labelled with either $1$ or $0$.}
\label{decisiontree}
\end{figure}

It is immediately obvious that binary decision trees are not an efficient method of representing boolean functions and propositional sentences, due to the high degree of redundancy in the produced tree. The storage requirements of such a data structure can immediately be significantly reduced by using a directed acyclic graph instead of a tree, and having identical subtrees of a graph be shared so that only a single copy needs to be stored. Since there are only two possible distinct leaf nodes, and with $n$ propositional letters the binary decision tree has $2^n$ leaf nodes, this saves at least $2^n-2$ nodes just on the last layer of the tree, but in practice can often also save many more on higher layers, especially if the propositional constraints are known to have some structure. Figure \ref{decisiondiagram} shows a representation in this form of the same propositional sentence as represented in Figure \ref{decisiontree}.

In the shown example the sharing of identical subtrees has reduced the total number of nodes to be stored by seven out of a total of fifteen. Despite this, any algorithm operating recursively on the diagram, that is at any point only able to observe a single node and carry out some operation on one or both of its child nodes will observe the exact same structure in both the tree and the directed acyclic graph.

\begin{figure}[tbh]
\centering
\includegraphics{{figs/diagram.dot}.pdf}
\caption{A decision diagram representing the same propositional sentence as Figure \ref{decisiontree} but with all identical subtrees deduplicated. The same conventions on arcs and node shapes are used.}
\label{decisiondiagram}
\end{figure}

However the shown decision diagram is still not maximally efficient, since every internal node is explicitly labelled with a propositional letter but it is possible determine the correct propositional letter based just on the depth of the node within the diagram. Instead of removing the labels and calculating them dynamically, binary decision diagrams (BDDs) use the labels to permit further compression of the graph. The most common compression strategy used is to omit any nodes which have $T$ and $F$ arcs out of them pointing to the same child node. A schematic representation of this compression strategy is shown in Figure \ref{bddcompression}. Compressing the decision diagram shown in \ref{decisiondiagram} in this way results in the diagram shown in \ref{bdd}, which decreases the number of stored nodes further by one.

\begin{figure}[tbh]
\centering

\begin{subfigure}[b]{0.49\textwidth}
\centering
\includegraphics{{figs/bddprecompression.dot}.pdf}
\caption{}
\label{bddprecompression}
\end{subfigure}
\begin{subfigure}[b]{0.49\textwidth}
\centering
\includegraphics{{figs/bddpostcompression.dot}.pdf}
\caption{}
\label{bddpostcompression}
\end{subfigure}

\caption{A schematic representation of BDD compression. Any part of a decision diagram that matches the pattern in \subref{bddprecompression} (where the box labelled BDD is any diagram) is replaced by \subref{bddpostcompression} (with the box replaced by whatever value it had in \subref{bddprecompression}).}
\label{bddcompression}
\end{figure}

\begin{figure}[tbh]
\centering
\includegraphics{{figs/bdd.dot}.pdf}
\caption{A BDD produced by applying a standard compression strategy to the diagram shown in Figure \ref{decisiondiagram}.}
\label{bdd}
\end{figure}

\section{Zero-Suppressed Binary Decision Diagrams}
Zero-Suppressed Binary Decision Diagrams (ZDDs) are binary decision diagrams that have been compressed using an alternative strategy. Nodes are omitted, and replaced by the node on their $F$ arc, if their $T$ arc points to a leaf node labelled $0$. This is intended to minimise the size of the decision diagram that needs to be stored particularly when the propositional sentence being represented is sparse, where "sparse" means that the propositional sentence has few satisfying assignments, and those satisfying assignments have few asserted variables. A schematic representation of this compression strategy is shown in Figure \ref{zddcompression}. If the diagram shown in Figure \ref{decisiondiagram} is instead compressed using this strategy, the resulting diagram is the one shown in Figure \ref{zdd}. Storing this diagram in memory requires three fewer nodes than the uncompressed decision diagram and two fewer than the BDD compression strategy. 

\begin{figure}[tbh]
\centering

\begin{subfigure}[b]{0.49\textwidth}
\centering
\includegraphics{{figs/zddprecompression.dot}.pdf}
\caption{}
\label{zddprecompression}
\end{subfigure}
\begin{subfigure}[b]{0.49\textwidth}
\centering
\includegraphics{{figs/zddpostcompression.dot}.pdf}
\caption{}
\label{zddpostcompression}
\end{subfigure}

\caption{A schematic representation of ZDD compression. Any part of a decision diagram that matches the pattern in \subref{zddprecompression} (where the box labelled ZDD is any part of the diagram) is replaced by \subref{zddpostcompression} (with the box replaced by whatever value it had in \subref{zddprecompression}).}
\label{zddcompression}
\end{figure}

\begin{figure}[tbh]
\centering
\includegraphics{{figs/zdd.dot}.pdf}
\caption{A ZDD produced by applying the ZDD compression strategy to the diagram shown in Figure \ref{decisiondiagram}.}
\label{zdd}
\end{figure}

\section{Equivalence}
Building a ZDD corresponding to the set of constraints mentioned in Section \ref{hamiltonianpath} and recursively computing the minimum distance path encoded by every subtree results in an algorithm for solving the travelling salesman problem that I expect to be equivalent to the dynamic programming solution described in Section \ref{dynamicprogramming}. I expect this to be true because the section of the graph below any given node corresponds to the set of possible paths for a particular set of nodes still to visit. This is the same set of values computed directly by the dynamic programming solution. The purpose of this project is to verify experimentally that both algorithms have the same run time complexity.

\chapter{Implementation}

\section{Technology Used}
All the code written for this project, the largest part of which is my implementation of a library for manipulating ZDDs, has been written in Python. A significant benefit of python is its native support for fast dictionaries, backed by hash maps, which I have used in several places throughout the library to cache the output of operations so they are not repeatedly recomputed and to reuse previously generated graph nodes whenever possible.  The \textit{dot} language for graph representation is used to allow ZDDs to be exported in a form which can be rendered easily using standard tools provided by the Graphviz package. This was useful both for debugging issues with the library and understanding the structure of the graphs produced by various sets of constraints, so that programs operating on them could be written.

\chapter{Evaluation}

\chapter{Conclusion}

%%%%%%%%%%%%%%%%%%%%%%%%%%%%%%%%%%%%%%%%%%%%%%%%%%%%%%%%%%%%%%%%%%%%%
% the bibliography
\addcontentsline{toc}{chapter}{Bibliography}
\bibliography{refs}

%%%%%%%%%%%%%%%%%%%%%%%%%%%%%%%%%%%%%%%%%%%%%%%%%%%%%%%%%%%%%%%%%%%%%
% the appendices
\appendix

\chapter{Project Proposal}

% Note: this file can be compiled on its own, but is also included by
% diss.tex (using the docmute.sty package to ignore the preamble)
\documentclass[12pt,a4paper,twoside]{article}
\usepackage[pdfborder={0 0 0}]{hyperref}
\usepackage[margin=25mm]{geometry}
\usepackage{graphicx}
\usepackage{parskip}
\begin{document}

\begin{center}
\Large
Computer Science Tripos -- Part II -- Project Proposal
\LARGE
Representing solutions to the Travelling Salesman Problem using ZDDs\\[4mm]

\large
A.~Hammond, Corpus Christi College

Originator: A. Hammond

6 October 2016
\end{center}

\vspace{5mm}

\textbf{Project Supervisor:} Dr T. Griffin

\textbf{Director of Studies:} Dr D. Greaves

\textbf{Project Overseers:} Dr A. Beresford  \& Dr G. Wynskel

% Main document

\section*{Introduction}

Zero suppressed binary decision diagrams (ZDDs) are a variant of binary decision diagrams (BDDs, covered in Part 1B) that are optimised for representing functions that are almost everywhere zero. Like with BDDs there are efficient algorithms for computing the combination of two ZDDs by most logical functions.

ZDDs are consequently a good candidate for representing families of subsets since these can be represented by characteristic functions, which will be sparse so long as the families are.

If we represent the solutions to NP-complete problems as collections of boolean variables then we can find valid solutions as 1-paths in a ZDD representing the constraints on those variables, and measure properties of those solutions efficiently by computing the property at each node of the ZDD. ZDDs have been used in this capacity in several recent pieces of research\cite{numberlink}\cite{ind}.

ZDDs have also been applied to operations on sets of paths in a graph. Paths are represented as a subset of a universe of variables, each of which corresponds to a particular node of the graph being visited at a particular index along the path. The most well known algorithm taking advantage of this is Donald Knuth's \textit{Simpath} algorithm, published in Volume 4A of \textit{The Art of Computer Programming}. 

Representing the set of paths that visit all nodes of a weighted complete graph in this way, and then recursively computing the minimum cost 1-path below each node allows us to compute the solution to the instance of the Travelling Salesman Problem described by that graph. I believe the complexity of this solution should be equivalent to an obvious Dynamic Programming solution. I expect there to be a trade off between the extra power introduced by using ZDDs (arbitrary additional constraints on the path could be added using standard ZDD operations) and the significant additional complexity of implementing the ZDD data structure.

\section*{Starting point}

The ZDD manipulation library used in this project will be built from scratch in order to explore the complexity of the data structure.

Current literature includes much analysis of ZDDs in general\cite{history}, but little on their use for solving instances of TSP.

\section*{Resources required}

I will primarily be using my personal laptop for this project running Linux Mint. All code will be backed up to github along with dissertation text in \LaTeX~form. Periodic backup will also be made to the MCS machines. In the event that my laptop becomes unusable for this project, all work will be possible to carry out using only MCS machines.

\section*{Work to be done}

The project breaks down into the following sub-projects:

\begin{enumerate}

\item Implementation of a generic ZDD data structure including functions to produce the primitive ZDDs used to build ZDDs that represent TSP instances and efficient functions for combining ZDDs.

\item Implementation of a solver for TSP using ZDD library.

\item Implementation of a dynamic programming solver for TSP.

\item Implementation of a brute force TSP solver for verifying the solutions of the other solvers on small problem instances (since a brute force solver should be easier to implement correctly and still be viable to run on instances with approximately 10 nodes).

\item Implementation a test harness to measure the efficiency of the two solutions on a number of random instances of TSP, including generating random instances of arbitrary size, and testing the correctness of the solutions. Testing might be achieved by comparing results among the three solvers, verifying by hand on problem instances small enough to make this viable and running the solvers on graphs where a minimum path can be found theoretically.

\item Experimental comparison of the performance (real time and asymptotic growth of operation count) of the three algorithms using the implemented test harness.

\end{enumerate}

\section*{Success criteria}

The project will be a success either if the ZDD based solution is shown experimentally to have equivalent complexity to the dynamic programming solution, or if this is experimentally refuted.

Evaluation of whether the two solutions have the same complexity will be achieved by plotting curves of run time and operation count of the two algorithms on problem sizes ranging to at least 30 nodes, linearly normalised to have the same value on the largest problem instance and the same average rate of growth. If the solutions have equivalent complexity we expect to see the same shape of graph for both, well approximated by a function in $\theta(n^22^n)$.

\section*{Possible extensions}

\begin{enumerate}

\item Demonstrate that the ZDD used to solve TSP can be built lazily and evaluate average performance improvement from doing so.

\item Compare the average performance of lazily built ZDDs to the known branch-and-bound algorithm for solving TSP.

\item Demonstrate that ZDDs can also be used to solve other NP-complete problems, for example graph colouring.

\end{enumerate}

\section*{Timetable}

Planned starting date is 22/10/2016.

\begin{enumerate}

\item \textbf{22nd October - 2nd November} Read ZDD section of Knuth's \textit{The Art of Computer Programming} to better understand state of the art, and select implementation programming language. Write first draft of dissertation background chapter.

\item \textbf{3rd November - 16th November} Build TSP generator and solution checker (possibly using brute force algorithm to verify solutions on small problems).

\item \textbf{17th November - 7th December} Build general ZDD implementation with functionality for combining multiple ZDDs according to standard boolean operators. Begin dissertation implementation chapter.

\item \textbf{8th December - 28th December} Implement dynamic programming algorithm for solving TSP.

\item \textbf{29th December - 18th January} Implement ZDD based algorithm on top of general ZDD implementation.

\item \textbf{19th January - 1st February} Verify correctness of both solutions and write progress report. Conclude first draft of dissertation implementation chapter. Write progress report presentation, hopefully covering testing of the solvers correctness.

\item \textbf{2nd February - 1st March} Performance analysis of both solutions, ideally determine validity of the hypothesis being tested. Write first draft of dissertation evaluation chapter.

\item \textbf{2nd March - 15th March} Second draft of main chapters of dissertation and evaluating extension feasibility.

\item \textbf{16th March - 5th April} Complete extensions if feasible, otherwise perform additional verification of evaluation.

\item \textbf{6th April - 25th April} Produce complete dissertation draft for start of full term to allow time for review by supervisor and overseers.

\item \textbf{26th April - 10th May}  Complete dissertation, performing further evaluation if necessary.

\item \textbf{11th May - 17th May} Submit dissertation.

\end{enumerate}

\begin{thebibliography}{9}

\bibitem{numberlink}
Yoshinaka, Ryo, et al. "Finding all solutions and instances of numberlink and slitherlink by ZDDs." Algorithms 5.2 (2012): 176-213.

\bibitem{ind}
Morrison, David R., Edward C. Sewell, and Sheldon H. Jacobson. "Characteristics of the maximal independent set ZDD." Journal of Combinatorial Optimization 28.1 (2014): 121-139.

\bibitem{history}
Minato, Shin-ichi. "Techniques of BDD/ZDD: brief history and recent activity." IEICE TRANSACTIONS on Information and Systems 96.7 (2013): 1419-1429.

\end{thebibliography}

\end{document}


\end{document}
